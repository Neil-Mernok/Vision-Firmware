\hypertarget{hwclock_8h}{
\section{include/hwclock.h File Reference}
\label{hwclock_8h}\index{include/hwclock.h@{include/hwclock.h}}
}
Timer support for AVR. 

{\tt \#include \char`\"{}config.h\char`\"{}}\par
{\tt \#include \char`\"{}ntrxtypes.h\char`\"{}}\par
\subsection*{Defines}
\begin{CompactItemize}
\item 
\#define \hyperlink{hwclock_8h_ee3c099580c48d099307b9493696fd23}{NKEYS}~4
\end{CompactItemize}
\subsection*{Functions}
\begin{CompactItemize}
\item 
uint32\_\-t \hyperlink{hwclock_8h_20623b49c1ed1dd212ad55448a651a9c}{hwclock} (void)
\begin{CompactList}\small\item\em return system clock in milliseconds \item\end{CompactList}\item 
void \hyperlink{hwclock_8h_3f8fa222f4976a01d7517c4ab86e7c4b}{hwclock\_\-init} (void)
\begin{CompactList}\small\item\em clock tick initialization \item\end{CompactList}\item 
void \hyperlink{hwclock_8h_033d55843d0da3c641301ad41a423fa6}{hwclock\_\-stop} (void)
\item 
void \hyperlink{hwclock_8h_369bb91cd7b5ed9cc814bfb036fcfe9d}{hwclock\_\-tune} (int8\_\-t tuning\-Direction)
\begin{CompactList}\small\item\em Modifies the time interval between any two subsequent timer ticks. \item\end{CompactList}\item 
void \hyperlink{hwclock_8h_c8c1a9f4fefba19cbaa462744963a5f3}{hwdelay} (uint32\_\-t t)
\item 
void \hyperlink{hwclock_8h_87f07be0d623932db45b4998f9175e8a}{HWDelayms} (uint16\_\-t ms)
\begin{CompactList}\small\item\em Delay processing for n milliseconds. \item\end{CompactList}\item 
void \hyperlink{hwclock_8h_5f99322ada800b90a8a06d0127b07513}{HWDelayus} (uint16\_\-t us)
\begin{CompactList}\small\item\em Delay processing for n microseconds. \item\end{CompactList}\item 
void \hyperlink{hwclock_8h_5046a51cac69ae907ee73317c3f12286}{Start\-Timer2} (void)
\end{CompactItemize}


\subsection{Detailed Description}
Timer support for AVR. 

\begin{Desc}
\item[Date:]2007-Dez-11 \end{Desc}
\begin{Desc}
\item[Author:]S. Radtke {\tt }(C) 2007 Nanotron Technologies\end{Desc}
\begin{Desc}
\item[Note:]Build\-Number = \char`\"{}Build\-Number : 7951\char`\"{};

This file contains the function definitions for the utility functions of the AVR hardware timer. \end{Desc}


Definition in file \hyperlink{hwclock_8h-source}{hwclock.h}.

\subsection{Define Documentation}
\hypertarget{hwclock_8h_ee3c099580c48d099307b9493696fd23}{
\index{hwclock.h@{hwclock.h}!NKEYS@{NKEYS}}
\index{NKEYS@{NKEYS}!hwclock.h@{hwclock.h}}
\subsubsection[NKEYS]{\setlength{\rightskip}{0pt plus 5cm}\#define NKEYS~4}}
\label{hwclock_8h_ee3c099580c48d099307b9493696fd23}




Definition at line 18 of file hwclock.h.

\subsection{Function Documentation}
\hypertarget{hwclock_8h_20623b49c1ed1dd212ad55448a651a9c}{
\index{hwclock.h@{hwclock.h}!hwclock@{hwclock}}
\index{hwclock@{hwclock}!hwclock.h@{hwclock.h}}
\subsubsection[hwclock]{\setlength{\rightskip}{0pt plus 5cm}uint32\_\-t hwclock (void)}}
\label{hwclock_8h_20623b49c1ed1dd212ad55448a651a9c}


return system clock in milliseconds 

\begin{Desc}
\item[Returns:]time in milliseconds\end{Desc}
This function returns the elapsed time since program start in milliseconds. 

Definition at line 223 of file hwclock.c.\hypertarget{hwclock_8h_3f8fa222f4976a01d7517c4ab86e7c4b}{
\index{hwclock.h@{hwclock.h}!hwclock_init@{hwclock\_\-init}}
\index{hwclock_init@{hwclock\_\-init}!hwclock.h@{hwclock.h}}
\subsubsection[hwclock\_\-init]{\setlength{\rightskip}{0pt plus 5cm}void hwclock\_\-init (void)}}
\label{hwclock_8h_3f8fa222f4976a01d7517c4ab86e7c4b}


clock tick initialization 

This function initializes the Timer 0 in the AVR to generate an interrupt ever 10 ms. 

Definition at line 196 of file hwclock.c.\hypertarget{hwclock_8h_033d55843d0da3c641301ad41a423fa6}{
\index{hwclock.h@{hwclock.h}!hwclock_stop@{hwclock\_\-stop}}
\index{hwclock_stop@{hwclock\_\-stop}!hwclock.h@{hwclock.h}}
\subsubsection[hwclock\_\-stop]{\setlength{\rightskip}{0pt plus 5cm}void hwclock\_\-stop (void)}}
\label{hwclock_8h_033d55843d0da3c641301ad41a423fa6}


\hypertarget{hwclock_8h_369bb91cd7b5ed9cc814bfb036fcfe9d}{
\index{hwclock.h@{hwclock.h}!hwclock_tune@{hwclock\_\-tune}}
\index{hwclock_tune@{hwclock\_\-tune}!hwclock.h@{hwclock.h}}
\subsubsection[hwclock\_\-tune]{\setlength{\rightskip}{0pt plus 5cm}void hwclock\_\-tune (int8\_\-t {\em tuning\-Direction})}}
\label{hwclock_8h_369bb91cd7b5ed9cc814bfb036fcfe9d}


Modifies the time interval between any two subsequent timer ticks. 

\begin{Desc}
\item[Parameters:]
\begin{description}
\item[{\em tuning\-Direction}]+1 to speed up the hwclock, -1 to speed down, 0 to reset to default. \end{description}
\end{Desc}


Definition at line 241 of file hwclock.c.\hypertarget{hwclock_8h_c8c1a9f4fefba19cbaa462744963a5f3}{
\index{hwclock.h@{hwclock.h}!hwdelay@{hwdelay}}
\index{hwdelay@{hwdelay}!hwclock.h@{hwclock.h}}
\subsubsection[hwdelay]{\setlength{\rightskip}{0pt plus 5cm}void hwdelay (uint32\_\-t {\em t})}}
\label{hwclock_8h_c8c1a9f4fefba19cbaa462744963a5f3}


Deprecated function please dont use for new projects 

Definition at line 66 of file hwclock.c.\hypertarget{hwclock_8h_87f07be0d623932db45b4998f9175e8a}{
\index{hwclock.h@{hwclock.h}!HWDelayms@{HWDelayms}}
\index{HWDelayms@{HWDelayms}!hwclock.h@{hwclock.h}}
\subsubsection[HWDelayms]{\setlength{\rightskip}{0pt plus 5cm}void HWDelayms (uint16\_\-t {\em ms})}}
\label{hwclock_8h_87f07be0d623932db45b4998f9175e8a}


Delay processing for n milliseconds. 

\begin{Desc}
\item[Parameters:]
\begin{description}
\item[{\em ms}]this is the delay in milliseconds\end{description}
\end{Desc}
This function is used for waiting TIMER\_\-RELOAD\_\-VALUEbefore continue with programm execution. Interrupts are still processed. 

Definition at line 134 of file hwclock.c.\hypertarget{hwclock_8h_5f99322ada800b90a8a06d0127b07513}{
\index{hwclock.h@{hwclock.h}!HWDelayus@{HWDelayus}}
\index{HWDelayus@{HWDelayus}!hwclock.h@{hwclock.h}}
\subsubsection[HWDelayus]{\setlength{\rightskip}{0pt plus 5cm}void HWDelayus (uint16\_\-t {\em us})}}
\label{hwclock_8h_5f99322ada800b90a8a06d0127b07513}


Delay processing for n microseconds. 

\begin{Desc}
\item[Parameters:]
\begin{description}
\item[{\em us}]this is the delay in microseconds\end{description}
\end{Desc}
This function is used for waiting before continue with programm execution. Interrupts are still processed. Because of the high inaccuracy of the delay function this function tries to compensate the delay error by adding an offset. 

Definition at line 89 of file hwclock.c.\hypertarget{hwclock_8h_5046a51cac69ae907ee73317c3f12286}{
\index{hwclock.h@{hwclock.h}!StartTimer2@{StartTimer2}}
\index{StartTimer2@{StartTimer2}!hwclock.h@{hwclock.h}}
\subsubsection[StartTimer2]{\setlength{\rightskip}{0pt plus 5cm}void Start\-Timer2 (void)}}
\label{hwclock_8h_5046a51cac69ae907ee73317c3f12286}


